\section{相关工作与理论基础}

\subsection{LaTeX排版系统简介}
\LaTeX{}是一种基于TeX的排版系统,由美国计算机科学家Leslie Lamport在20世纪80年代初期开发。与其他所见即所得的文字处理软件不同,\LaTeX{}采用标记语言的形式,用户通过编写带有标记的文本文件,由\LaTeX{}引擎将其转换为高质量的排版文档\cite{kopka2003guide}。\LaTeX{}的核心理念是分离内容与格式,使作者能够专注于文档的内容创作,而将排版的细节交由系统处理。

\LaTeX{}系统的主要优势包括:

\begin{enumerate}
    \item 优秀的数学公式排版能力,能够处理复杂的数学表达式;
    \item 自动化的交叉引用、目录生成和参考文献管理;
    \item 一致的格式控制和精确的排版质量;
    \item 对大型文档的良好支持和模块化管理;
    \item 开源免费,具有强大的扩展性。
\end{enumerate}

这些特性使得\LaTeX{}成为科学研究、学术出版领域的首选排版工具,特别是在数学、物理、计算机科学等学科领域被广泛使用。

\subsection{国内外学位论文LaTeX模版现状}
随着\LaTeX{}在学术界的普及,许多高校和研究机构已经开发了符合各自规范的学位论文\LaTeX{}模版。在国际上,如麻省理工学院(MIT)、斯坦福大学、剑桥大学等知名高校均提供官方支持的\LaTeX{}论文模版;在国内,清华大学、北京大学、中国科学院大学等高校也已开发了成熟的\LaTeX{}学位论文模版。

这些模版的共同特点是:
\begin{enumerate}
    \item 严格遵循学校制定的学位论文格式规范;
    \item 采用模块化的设计,便于维护和使用;
    \item 提供丰富的宏包和命令,简化常见的排版任务;
    \item 配备详细的使用文档和示例。
\end{enumerate}

然而,曲阜师范大学目前尚未有官方支持的\LaTeX{}学位论文模版,这给希望使用\LaTeX{}撰写学位论文的研究生带来了不便。

\subsection{曲阜师范大学学位论文格式规范}
曲阜师范大学研究生学位论文撰写规范对论文的格式有严格要求,主要包括以下几个方面:

\subsubsection{页面设置}
论文采用A4纸(210mm $\times$ 297mm)打印,页边距为:上2.5cm、下2.5cm、左3.0cm、右2.5cm。正文行距设置为20磅,段落首行缩进2个汉字。

\subsubsection{字体与字号}
论文中文部分主要使用宋体、黑体、楷体等字体,英文部分使用Times New Roman字体。不同层级标题采用不同的字号和字体:一级标题使用三号黑体,二级标题使用四号黑体,三级标题使用小四号黑体,正文内容使用小四号宋体。

\subsubsection{论文结构}
学位论文一般包括封面、中英文摘要、目录、正文、参考文献、附录、致谢等部分。每个部分都有特定的格式要求,如摘要字数限制、关键词数量、目录生成规则等。

\subsubsection{图表与公式}
论文中的图表需要有规范的编号和标题,图标题置于图下方,表标题置于表上方。公式应居中排版,并在右侧给出编号。

\subsubsection{参考文献}
参考文献的著录格式应符合国家标准GB/T 7714-2015《信息与文献 参考文献著录规则》的规定,按照在正文中出现的先后顺序在文末列出。

本模版的开发将严格遵循上述格式规范,确保生成的论文符合学校要求。
