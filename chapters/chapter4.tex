\section{使用指南}

本章将详细介绍模版的安装和使用方法,帮助用户快速上手,顺利完成学位论文的撰写。

\subsection{环境准备}
使用本模版前,需要在计算机上安装\LaTeX{}编译环境。推荐安装以下软件:

\begin{enumerate}
    \item \textbf{\TeX{}发行版}:建议安装TeX Live(跨平台)、MiKTeX(Windows)或MacTeX(macOS)。这些发行版包含了\LaTeX{}编译器和常用宏包。
    \item \textbf{编辑器}:推荐使用TeXstudio、VS Code(配合LaTeX Workshop插件)或Overleaf(在线平台)等编辑器,它们提供语法高亮、自动补全等功能,提高编辑效率。
    \item \textbf{参考文献管理工具}:如JabRef、Zotero等,用于管理BibTeX格式的参考文献数据库。
\end{enumerate}

\subsection{模版获取与安装}
获取模版的方式有两种:
\begin{enumerate}
    \item 从GitHub仓库下载:访问\url{https://github.com/wtzmx/QFNU-Thesis-LaTeX},点击"Code"按钮,选择"Download ZIP"下载压缩包。
    \item 使用Git克隆:在命令行中执行Git克隆命令,地址为 \url{https://github.com/wtzmx/QFNU-Thesis-LaTeX}
\end{enumerate}

下载完成后,解压文件(如使用ZIP下载方式),即可获得完整的模版文件结构。

\subsection{基本使用流程}

\subsubsection{编辑论文信息}
首先需要在styles/info.tex文件中设置论文的基本信息,包括论文标题、作者姓名、学号、专业、导师信息等。例如:设置论文标题为"基于LaTeX的曲阜师范大学硕博论文模版",作者为"张三",学号为"202001010101",专业为"计算机科学与技术",导师为"李四 教授",论文完成日期为"2023年6月"。另外,还需设置中英文关键词,如中文关键词可设为"LaTeX; 论文模版; 曲阜师范大学; 排版系统",英文关键词为"LaTeX; Thesis Template; Qufu Normal University; Typesetting System"。

\subsubsection{编写摘要}
在front/abstract.tex文件中编写中英文摘要。中文摘要使用cnabstract环境,内容可以是"本文介绍了一种基于LaTeX的曲阜师范大学硕博学位论文模版..."。英文摘要使用enabstract环境,内容可以是"This paper introduces a LaTeX-based thesis template for Qufu Normal University..."。

\subsubsection{编写正文内容}
在chapters目录下的章节文件中编写正文内容。每个章节使用\verb|\section|命令创建一级标题,\verb|\subsection|创建二级标题,\verb|\subsubsection|创建三级标题。例如,在chapter1.tex中可以创建"绪论"一级标题,并在其下创建"研究背景"和"研究意义"等二级标题,然后填写相应内容。

\subsubsection{插入图片}
使用graphicx宏包提供的\verb|\includegraphics|命令插入图片。插入图片时,通常将图片放在figure环境中,设置位置选项\verb|[htbp]|,使用\verb|\centering|命令使图片居中,\verb|\includegraphics|命令控制图片大小(如\verb|width=0.8\textwidth|),\verb|\caption|命令添加图片标题,\verb|\label|命令为图片添加标签用于后续引用,如\ref{fig:example}。

\begin{figure}[htbp]
    \centering
    \includegraphics[width=0.8\textwidth]{figure/logo/logo-1.png}
    \caption{示例图片}
    \label{fig:example}
\end{figure}

\subsubsection{创建表格}
使用tabular环境创建表格。创建表格时,通常将表格放在table环境中,设置位置选项\verb|[htbp]|,使用\verb|\centering|命令使表格居中,\verb|\caption|命令添加表格标题,\verb|\label|命令为表格添加标签,如\ref{tab:example}。在tabular环境中,使用\verb|{ccc}|等参数设置列对齐方式,使用\verb|\toprule|、\verb|\midrule|和\verb|\bottomrule|命令添加表格线,用\verb|&|分隔单元格,用\verb|\\|换行。

\begin{table}[htbp]
    \centering
    \caption{示例表格}
    \label{tab:example}
    \begin{tabular}{ccc}
        \toprule    
        \textbf{列1} & \textbf{列2} & \textbf{列3} \\
        \midrule
        \textbf{行1} & 1 & 2 \\
        \textbf{行2} & 3 & 4 \\
        \bottomrule
    \end{tabular}   
\end{table}

\subsubsection{插入公式}
使用amsmath宏包提供的数学环境插入公式。行内公式使用\verb|$...$|格式,如\verb|$E=mc^2$|。行间公式使用\verb|equation|环境,如\verb|F = ma|,并可用\verb|\label|命令为公式添加标签,如\ref{eq:example}。

\begin{equation}
    \label{eq:example}
    F = ma
\end{equation}

\subsubsection{参考文献管理}
在bib/references.bib文件中添加参考文献条目,格式为BibTeX格式。例如,可以添加书籍条目(包含\verb|title|、\verb|author|、\verb|year|、\verb|publisher|等字段)和文章条目(包含\verb|title|、\verb|author|、\verb|journal|、\verb|year|等字段)。在正文中使用\verb|\cite|命令引用参考文献,如\cite{kopka2003guide}。

\subsection{编译方法}
编译论文需要按照以下步骤:

\begin{enumerate}
    \item 运行xelatex编译main.tex文件,生成辅助文件
    \item 运行bibtex编译.aux文件,处理参考文献
    \item 再次运行xelatex编译main.tex文件,更新引用
    \item 最后再运行一次xelatex,确保所有交叉引用正确
\end{enumerate}

在TeXstudio等编辑器中,可以使用"编译"按钮执行上述步骤。使用命令行时,可以依次执行xelatex main、bibtex main、xelatex main、xelatex main这四个命令。

\subsection{常见问题与解决方案}

\subsubsection{中文字体问题}
问题:编译时出现"找不到字体"的错误。

解决方案:确保字体目录fonts/simsun.ttf存在,并设置为默认字体。

\subsubsection{图片路径问题}
问题:插入图片时提示"找不到文件"。

解决方案:确保图片文件放在正确的位置(figures目录下),并使用正确的相对路径。图片格式推荐使用PDF、PNG或JPG。

\subsubsection{参考文献编译问题}
问题:参考文献无法显示或格式不正确。

解决方案:检查bib文件格式是否正确,确保运行了bibtex编译。如果使用国标格式,需要安装gbt7714宏包。

\subsubsection{编译速度问题}
问题:编译速度较慢,特别是有大量图片时。

解决方案:可以在最终版本完成前注释掉\verb|\includeonly|命令,只编译正在编辑的章节,加快编译速度。

通过本章的指南,用户可以快速掌握模版的使用方法,顺利完成学位论文的撰写工作。在使用过程中遇到的具体问题,可以参考附录中的常见问题解答,或通过GitHub提交Issue获取帮助。
