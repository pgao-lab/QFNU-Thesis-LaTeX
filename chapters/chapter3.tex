\section{模版设计与实现}

\subsection{整体设计思路}
本模版的设计遵循了以下几个核心原则:

\begin{enumerate}
    \item \textbf{严格符合规范}:严格按照曲阜师范大学学位论文撰写规范进行设计,确保生成的论文满足学校要求。
    \item \textbf{模块化设计}:将模版划分为多个功能模块,便于维护和扩展。
    \item \textbf{用户友好}:简化用户使用流程,提供清晰的文档结构和注释,降低学习成本。
    \item \textbf{兼容性考虑}:确保模版在主流的\LaTeX{}发行版(如TeX Live、MiKTeX等)上能够正常编译。
\end{enumerate}

基于上述原则,模版采用了类似封装的设计方法,将复杂的格式设置封装在样式文件中,用户只需关注内容的创作,无需深入了解\LaTeX{}的复杂命令。

\subsection{文件结构组织}
本模版的文件结构清晰,主要包括以下几个部分:

\begin{itemize}
    \item \textbf{main.tex}:主文件,控制整体结构,用户通过此文件编译整个论文。
    \item \textbf{styles目录}:包含各种样式文件,如宏包导入(packages.tex)、格式设置(format.tex)和论文信息设置(info.tex)。
    \item \textbf{front目录}:前置部分,包括封面设计(cover.tex)、中英文摘要(abstract.tex)和目录设置(tableofcontents.tex)。
    \item \textbf{chapters目录}:各章节内容,用户在此编写论文的主体部分。
    \item \textbf{bib目录}:存放参考文献数据库文件(references.bib)。
    \item \textbf{figures目录}:用于存放论文中使用的图片。
\end{itemize}

这种结构设计具有以下优点:
\begin{enumerate}
    \item 各部分内容分离,便于单独编辑和管理;
    \item 样式与内容分离,用户可以专注于内容撰写;
    \item 模块化组织,便于后期维护和扩展。
\end{enumerate}

\subsection{核心功能实现}

\subsubsection{文档类与宏包选择}
本模版基于标准的article文档类进行开发,并根据需要导入了一系列宏包,主要包括:ctex(中文支持)、geometry(页面设置)、fancyhdr(页眉页脚)、graphicx(图片支持)、booktabs(表格美化)、amsmath(数学公式)、hyperref(超链接)和natbib(参考文献)等。

其中,ctex宏包提供了对中文的支持,geometry宏包用于设置页面尺寸和边距,fancyhdr宏包用于定制页眉页脚。这些宏包的选择充分考虑了中文学术论文的特殊需求,确保生成的文档符合规范要求。

\subsubsection{格式设置}
为了符合学校的格式要求,模版在format.tex文件中对文档格式进行了详细设置,包括:

\begin{itemize}
    \item \textbf{页面设置}:A4纸张,上下页边距2.5cm,左边距3.0cm,右边距2.5cm。
    \item \textbf{行距设置}:采用1.5倍行距,符合学校规范。
    \item \textbf{章节标题格式}:一级标题(section)使用三号黑体居中,二级标题(subsection)使用四号黑体,三级标题(subsubsection)使用小四号黑体。
\end{itemize}

\subsubsection{封面设计}
封面是论文的重要组成部分,模版通过在cover.tex文件中定义专门的命令实现了封面生成。封面包含以下主要元素:

\begin{itemize}
    \item 学校徽标(居中)
    \item 论文题目(二号黑体加粗,居中)
    \item 学生信息(学号、专业、姓名和导师,三号字体,表格形式)
    \item 日期(四号字体,页面底部居中)
\end{itemize}

用户只需在info.tex文件中填写相关信息,模版会自动生成符合要求的封面。

\subsubsection{中英文摘要实现}
摘要部分需要同时包含中英文内容,模版通过定义专门的环境实现了规范的摘要格式:

\begin{itemize}
    \item \textbf{中文摘要}:标题为三号黑体"摘要",正文为小四号宋体,关键词部分小五号,其中"关键词"三字使用黑体。
    \item \textbf{英文摘要}:标题为三号"Abstract",正文为小四号Times New Roman,关键词部分小五号,其中"Keywords"使用加粗处理。
\end{itemize}

这些摘要环境自动添加到目录中,并使用标准格式显示,极大方便了用户使用。

\subsubsection{参考文献管理}
本模版采用BibTeX进行参考文献管理,配置了符合国标(GB/T 7714-2015)的参考文献格式。采用数字标注方式,引用时使用方括号上标形式。参考文献按照引用顺序在文末列出,且排版格式严格符合国家标准要求。

\subsubsection{图表环境定制}
为了满足学校对图表格式的要求,模版对图表环境进行了定制:

\begin{itemize}
    \item 图片标题改为"图",表格标题改为"表"
    \item 图表编号格式为"章节号-序号",如"图1-2"表示第1章第2个图
    \item 图标题置于图下方,表标题置于表上方
    \item 标题使用五号宋体,居中对齐
\end{itemize}

通过以上设计与实现,本模版能够生成符合曲阜师范大学学位论文格式要求的规范文档,为研究生撰写学位论文提供有力支持。
