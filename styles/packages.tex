%----------------------------------------------------------------------------------------
% 包导入文件 - 包含所有需要的LaTeX包
%----------------------------------------------------------------------------------------

%----------------------------------------------------------------------------------------
% 基础包导入
%----------------------------------------------------------------------------------------
\usepackage{fontspec}                        % 字体设置
\usepackage{xeCJK}                           % 中文字体支持
\usepackage[fontset=none]{ctex}             % 使用自定义字体设置
\usepackage{geometry}                        % 页面设置
\usepackage{graphicx}                        % 图片支持
\usepackage{float}                           % 浮动体控制
\usepackage{fancyhdr}                        % 页眉页脚设置

%----------------------------------------------------------------------------------------
% 字体和排版相关包
%----------------------------------------------------------------------------------------
\usepackage{CJKfntef}                        % 中文下划线支持
\usepackage{titlesec}                        % 章节标题格式
\usepackage{tocloft}                         % 自定义目录格式
\usepackage{csquotes}                        % 引号支持

%----------------------------------------------------------------------------------------
% 数学和符号相关包
%----------------------------------------------------------------------------------------
\usepackage{amsmath}                         % 数学公式
\usepackage{amssymb}                         % 数学符号
\usepackage{pifont}                          % 特殊符号

%----------------------------------------------------------------------------------------
% 表格和图片相关包
%----------------------------------------------------------------------------------------
\usepackage{booktabs}                        % 三线表格
\usepackage{multirow}                        % 表格多行合并
\usepackage{rotating}                        % 旋转文字
\usepackage{caption}                         % 图表标题格式
\usepackage[dvipsnames,svgnames,x11names]{xcolor}  % 颜色支持
\usepackage{colortbl}                        % 表格颜色
\usepackage{threeparttable}                  % 表格注释
\usepackage{tcolorbox}                       % 创建美观的彩色框
\tcbuselibrary{breakable}                    % 加载breakable库

%----------------------------------------------------------------------------------------
% 代码和verbatim环境相关包
%----------------------------------------------------------------------------------------
\usepackage{listings}                        % 代码列表
\usepackage{fancyvrb}                        % 增强的verbatim环境
\usepackage{courier}                         % Courier字体,用于代码

%----------------------------------------------------------------------------------------
% 算法和代码相关包
%----------------------------------------------------------------------------------------
\usepackage{algorithm}                       % 算法环境
\usepackage{algpseudocode}                   % 算法伪代码

%----------------------------------------------------------------------------------------
% 参考文献相关包
%----------------------------------------------------------------------------------------
\usepackage{gbt7714}                         % 中文参考文献格式
\bibliographystyle{gbt7714-numerical}        % 数字引用样式

%----------------------------------------------------------------------------------------
% 列表相关包
%----------------------------------------------------------------------------------------
\usepackage{enumitem}                        % 列表控制

%----------------------------------------------------------------------------------------
% 超链接设置
%----------------------------------------------------------------------------------------
\usepackage{hyperref}
\hypersetup{
    colorlinks=false,                       % 不使用彩色链接
    pdfborder={0 0 1},                      % 设置PDF边框宽度(恢复默认边框)
    linkcolor=blue,                         % 设置内部链接颜色
    filecolor=magenta,                      % 设置文件链接颜色
    urlcolor=cyan,                          % 设置URL链接颜色
    pdftitle={基于Latex的曲阜师范大学硕博论文模版},
    pdfauthor={王鹏}
}

% 定义可换行的下划线命令
\newcommand{\sounder}[1]{\CJKunderline{#1}} 