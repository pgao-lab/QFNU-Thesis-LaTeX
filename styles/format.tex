%----------------------------------------------------------------------------------------
% 格式设置文件 - 包含所有样式定义
%----------------------------------------------------------------------------------------

%----------------------------------------------------------------------------------------
% 字体设置
%----------------------------------------------------------------------------------------
% 中文字体设置 - 使用本地字体文件
\setCJKmainfont[
    Path = ./font/,
    AutoFakeSlant = 0.4,
    AutoFakeBold = 3.17
]{FangSong.ttf}

\setCJKsansfont[
    Path = ./font/,
    AutoFakeBold = 3.17
]{simhei.ttf}

\setCJKmonofont[
    Path = ./font/,
    AutoFakeBold = 3.17
]{KaiTi.ttf}

% 设置西文主字体为Times New Roman(使用本地字体文件)
\setmainfont[
    Path = ./font/,
    BoldFont = {Times New Roman Bold.ttf},
    ItalicFont = {Times New Roman Italic.ttf},
    BoldItalicFont = {Times New Roman Bold Italic.ttf}
]{Times New Roman.ttf}

% 所有西文字体统一使用Times New Roman
\setsansfont[
    Path = ./font/,
    BoldFont = {Times New Roman Bold.ttf},
    ItalicFont = {Times New Roman Italic.ttf},
    BoldItalicFont = {Times New Roman Bold Italic.ttf}
]{Times New Roman.ttf}

% 设置等宽字体为Courier New,用于代码显示
\setmonofont{Courier New}

% 定义英文标题字体
\newfontfamily\entitlefont[
    Path = ./font/,
    BoldFont = {Times New Roman Bold.ttf},
    ItalicFont = {Times New Roman Italic.ttf},
    BoldItalicFont = {Times New Roman Bold Italic.ttf}
]{Times New Roman.ttf}

% 中文字体族设置
\setCJKfamilyfont{zhsong}[
    Path = ./font/,
    AutoFakeBold = 3.17
]{SimSun.ttf}
\setCJKfamilyfont{zhhei}[
    Path = ./font/,
    AutoFakeBold = 3.17
]{simhei.ttf}
\setCJKfamilyfont{zhfs}[
    Path = ./font/,
    AutoFakeBold = 3.17
]{FangSong.ttf}
\setCJKfamilyfont{zhkai}[
    Path = ./font/,
    AutoFakeBold = 3.17
]{KaiTi.ttf}

% 中文字体命令
\newcommand{\songti}{\CJKfamily{zhsong}}    % 宋体
\newcommand{\heiti}{\CJKfamily{zhhei}}      % 黑体
\newcommand{\fangsong}{\CJKfamily{zhfs}}    % 仿宋
\newcommand{\kaishu}{\CJKfamily{zhkai}}     % 楷书

% 字体定义 - 因已经在上面定义过字体,此处保留但不再重新定义
\newfontfamily{\arial}{Arial}                % 定义Arial字体
\newfontfamily{\timesnewroman}[
    Path = ./font/,
    BoldFont = {Times New Roman Bold.ttf},
    ItalicFont = {Times New Roman Italic.ttf},
    BoldItalicFont = {Times New Roman Bold Italic.ttf}
]{Times New Roman.ttf}  % 定义Times New Roman字体

% 中文标点设置
\xeCJKsetup{
  AutoFakeBold=true,
  AutoFakeSlant=true,
  PunctStyle=quanjiao                        % 全角式标点
}
\punctstyle{kaiming}
\MakeOuterQuote{"}

%----------------------------------------------------------------------------------------
% 页面设置
%----------------------------------------------------------------------------------------
\geometry{
    top=2.54cm,
    bottom=2.54cm,
    left=3cm,
    right=3cm,
    headheight=13pt                          % 页眉高度
}  

% 添加全局断行设置
\tolerance=1000
\emergencystretch=\hsize

%----------------------------------------------------------------------------------------
% 目录格式设置
%----------------------------------------------------------------------------------------
% 设置目录点线
\renewcommand{\cftsecleader}{\cftdotfill{\cftdotsep}}  % 为section级别添加点线
\renewcommand{\cftdotsep}{0.8}               % 设置点的间距

%----------------------------------------------------------------------------------------
% 章节标题格式设置
%----------------------------------------------------------------------------------------
% 一级标题格式
\titleformat{\section}{\centering\heiti\zihao{-3}\bfseries}{第\arabic{section}章}{1em}{}
\titlespacing{\section}{0pt}{24pt}{18pt}     % 设置章节标题的间距

% 二级标题格式
\titleformat{\subsection}{\heiti\zihao{4}}{\arabic{section}.\arabic{subsection}}{1em}{}
\titlespacing{\subsection}{0pt}{10pt}{10pt}  % 设置二级标题的间距

% 三级标题格式
\titleformat{\subsubsection}{\heiti\zihao{-4}\bfseries}{\arabic{section}.\arabic{subsection}.\arabic{subsubsection}}{1em}{}
\titlespacing{\subsubsection}{0pt}{10pt}{10pt}  % 设置三级标题的间距

% 四级标题格式
\titleformat{\paragraph}
  {\songti\zihao{-4}\bfseries}               % 格式:宋体,小四号,加粗
  {}                                         % 无标签
  {0em}                                      % 标签与标题间距
  {(\arabic{paragraph})}                   % 使用阿拉伯数字,带中文括号
  []                                         % 后缀为空
\titlespacing{\paragraph}{0pt}{12pt}{0em}    % 段落标题间距

% 确保段落计数器正确工作
\setcounter{secnumdepth}{4}                  % 允许到第四级标题的编号
\counterwithin{paragraph}{subsubsection}     % 段落计数器依赖于三级标题

%----------------------------------------------------------------------------------------
% 设置列表环境参数
%----------------------------------------------------------------------------------------
\setlength{\leftmargini}{2em}               % 一级列表缩进
\setlength{\leftmarginii}{2em}              % 二级列表缩进
\setlength{\itemindent}{0em}                % 项目缩进
\setlength{\labelsep}{0.5em}                % 标签和文本间距
\setlength{\labelwidth}{2em}                % 标签宽度

% 列表格式设置
\setlist{nosep}                             % 移除所有列表间距
\setlist[enumerate,1]{label=\arabic{enumi}.}  % 一级编号格式
\setlist[enumerate,2]{label=(\arabic{enumii})}  % 二级编号格式
\setlist[itemize,1]{label={}}               % 移除一级itemize的圆点

%----------------------------------------------------------------------------------------
% 图表标题格式设置
%----------------------------------------------------------------------------------------
% 图表按章节编号设置
\numberwithin{figure}{section}              % 将图片计数器与章节关联
\renewcommand{\thefigure}{\thesection-\arabic{figure}}  % 设置图片编号格式为(章节号-图片号)
\numberwithin{table}{section}               % 将表格计数器与章节关联
\renewcommand{\thetable}{\thesection-\arabic{table}}    % 设置表格编号格式为(章节号-表格号)

% 公式按章节编号设置
\numberwithin{equation}{section}           % 将公式计数器与章节关联
\renewcommand{\theequation}{\thesection-\arabic{equation}}  % 设置公式编号格式为(章节号-公式号)

% 自定义标题字体
\DeclareCaptionFont{myfont}{\songti\zihao{5}}

% 图片标题格式
\captionsetup[figure]{
    font=myfont,                             % 使用自定义字体
    name=图,                                % 确保标题前缀为"图"
    justification=centering,                 % 居中对齐
    skip=5pt,                               % 段后间距
    belowskip=0pt                            % 段前间距
}

% 表格标题格式
\captionsetup[table]{
    font=myfont,                             % 使用自定义字体
    name=表,                                % 确保标题前缀为"表"
    justification=centering,                 % 居中对齐
    skip=0pt,                                % 段后间距
    belowskip=0pt,                          % 段前间距
    position=top                             % 标题在表格上方
}

%----------------------------------------------------------------------------------------
% verbatim环境和代码格式设置
%----------------------------------------------------------------------------------------
% 设置FancyVerbatim的默认格式
\fvset{
    fontfamily=courier,                     % 使用Courier字体族
    fontsize=\zihao{5},                     % 设置字号为5号
    numbers=left,                           % 行号在左侧
    numbersep=5pt,                          % 行号与代码间距
    frame=single,                           % 单线框
    framesep=5pt,                           % 框与内容间距
    commandchars=\\\{\},                    % 命令字符
    xleftmargin=15pt,                       % 左边距
    xrightmargin=0pt                        % 右边距
}

% 定义代码块环境
\DefineVerbatimEnvironment{codeblock}{Verbatim}{
    fontfamily=courier,
    fontsize=\zihao{5},
    numbers=left,
    frame=single,
    framesep=5pt,
    xleftmargin=15pt
}

% 使用listings宏包设置代码格式
\lstset{
    basicstyle=\ttfamily\zihao{5},          % 基本样式
    numbers=left,                           % 行号显示在左侧
    numberstyle=\tiny\color{gray},          % 行号样式
    stepnumber=1,                           % 行号步进
    numbersep=5pt,                          % 行号与代码的距离
    backgroundcolor=\color{white},          % 背景色
    showspaces=false,                       % 不显示空格
    showstringspaces=false,                 % 不显示字符串中的空格
    showtabs=false,                         % 不显示制表符
    frame=single,                           % 边框
    tabsize=2,                              % 制表符长度
    captionpos=b,                           % 标题位置
    breaklines=true,                        % 自动断行
    breakatwhitespace=false,                % 只在空格处断行
    keywordstyle=\color{blue},              % 关键字样式
    commentstyle=\color{olive},             % 注释样式
    stringstyle=\color{red},                % 字符串样式
    escapeinside={\%*}{*\%},                % 逃逸字符
    xleftmargin=15pt,                       % 左边距
    xrightmargin=0pt,                       % 右边距
    aboveskip=1em,                          % 上间距
    belowskip=1em                           % 下间距
}

% 设置普通verbatim环境格式
\RecustomVerbatimEnvironment{verbatim}{Verbatim}{
    fontfamily=courier,
    fontsize=\zihao{5},
    frame=single,
    framesep=5pt,
    xleftmargin=15pt
}

%----------------------------------------------------------------------------------------
% 参考文献格式设置
%----------------------------------------------------------------------------------------
% 设置参考文献的字体
\renewcommand{\bibfont}{\zihao{-4}\songti\timesnewroman}  % 设置基本字号,中文用宋体,英文用Times New Roman

%----------------------------------------------------------------------------------------
% 算法环境设置
%----------------------------------------------------------------------------------------
% 修改算法环境中的标题为中文
\floatname{algorithm}{算法}                          % 将 Algorithm 改为 算法

%----------------------------------------------------------------------------------------
% 自定义命令
%----------------------------------------------------------------------------------------
% 中文引号命令
\newcommand{\cquote}[1]{"\hskip 0pt #1\hskip 0pt "}

% 垂直排列字符
\makeatletter
\newcommand{\vertchar}[1]{%
    \@tfor\tmp:=#1\do{\tmp\\}%
}
\makeatother

% 添加这个新命令来处理无编号章节的页眉
\newcommand{\unnumberedsection}[1]{%
    \phantomsection
    \section*{#1}
    \addcontentsline{toc}{section}{#1}
    \markboth{#1}{}
}

% 定义对勾和叉号
\newcommand{\cmark}{\ding{51}}               % 对勾符号
\newcommand{\xmark}{\ding{55}}               % 叉号符号

%----------------------------------------------------------------------------------------
% 页眉页脚设置
%----------------------------------------------------------------------------------------
% 页眉页脚样式设置
\pagestyle{fancy}
\fancyhf{}                                   % 清除所有页眉页脚
\renewcommand{\headrulewidth}{0.5pt}         % 页眉横线宽度

% 定义正文页面样式
\newcommand{\mainpagestyle}{%
    \pagestyle{fancy}
    \fancyhf{}
    \renewcommand{\headrulewidth}{0.5pt}
    \fancyhead[C]{\zihao{5}\songti \leftmark}    % 页眉中间显示当前章节名称
    \fancyfoot[C]{\zihao{5}\rmfamily \thepage}   % 页脚中间显示页码
}

% 重新定义章节标记格式,使其适合页眉显示
\renewcommand{\sectionmark}[1]{\markboth{第\arabic{section}章\quad #1}{}}

% 定义前言页面样式(罗马数字页码)
\newcommand{\frontmatterpagestyle}{%
    \pagestyle{fancy}
    \fancyhf{}
    \renewcommand{\headrulewidth}{0.5pt}
    \fancyfoot[C]{\zihao{5}\rmfamily \thepage}   % 页脚中间显示页码
}

% 设置正文默认字体格式
\newcommand{\setmaintextstyle}{%
    \renewcommand{\normalsize}{\songti\zihao{-4}}  % 设置默认字体为宋体小四号
    \setlength{\baselineskip}{20pt}              % 设置行距
    \normalsize                                  % 激活设置
} 