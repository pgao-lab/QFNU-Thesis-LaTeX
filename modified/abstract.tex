%----------------------------------------------------------------------------------------
% 摘要部分
%----------------------------------------------------------------------------------------
% 设置页码格式为大写罗马数字,并从摘要页开始计数
\newpage
\setcounter{page}{1}
\pagenumbering{Roman}

% 设置中文摘要页眉和页码
\frontmatterpagestyle  % 使用前言样式
\fancyhead[C]{\zihao{5}\songti 摘要}     % 页眉中间显示"摘要"

% 中文摘要
\thispagestyle{fancy}
\phantomsection
\addcontentsline{toc}{section}{摘要}     % 添加中文摘要到目录
\begin{center}
    {\zihao{3}\heiti\bfseries           % 黑体三号加粗
    摘要
    }
\end{center}

{\zihao{-4}\songti                       % 宋体小四号
\setlength{\baselineskip}{20pt}          % 设置行距
本模版是基于\LaTeX{}开发的曲阜师范大学硕博论文模版,旨在为曲阜师范大学研究生撰写学位论文提供规范化、标准化的排版工具。模版严格遵循学校学位论文撰写规范,包括封面、中英文摘要、目录、正文、参考文献等各部分的格式要求。模版采用模块化设计,各部分内容分文件管理,便于维护和修改。通过本模版,研究生可以将精力集中在学术内容的创作上,而不必花费过多时间在格式调整上。本文详细介绍了模版的结构、使用方法以及常见问题的解决方案,为使用者提供全面的指导。


}

\vspace{1em}                             % 添加一些间距
{\zihao{-4}\songti                       % 宋体小四号
\setlength{\baselineskip}{20pt}          % 设置行距
\noindent\textbf{关键词:}\zhkeywords

% 英文摘要
\newpage
% 设置英文摘要页眉和页码
\frontmatterpagestyle  % 使用前言样式
\fancyhead[C]{\zihao{5}\timesnewroman Abstract}  % 页眉改为Times New Roman字体

% 英文摘要页
\phantomsection
\addcontentsline{toc}{section}{Abstract}  % 添加英文摘要到目录
\begin{center}
    {\zihao{3}\timesnewroman\bfseries     % 三号Times New Roman加粗
    Abstract
    }
\end{center}

{\zihao{-4}\selectfont\timesnewroman     % 小四号Times New Roman
\setlength{\baselineskip}{20pt}          % 设置行距
This template is a \LaTeX{}-based thesis template for Qufu Normal University's master and doctoral students, designed to provide a standardized typesetting tool for academic thesis writing. The template strictly adheres to the university's thesis formatting requirements, including cover page, Chinese and English abstracts, table of contents, main text, references, and other sections. With a modular design approach, all components are managed in separate files for easy maintenance and modification. Through this template, graduate students can focus their energy on creating academic content rather than spending excessive time on format adjustments. This paper provides detailed information about the template's structure, usage methods, and solutions to common problems, offering comprehensive guidance for users.
}

\vspace{1em}                             % 添加一些间距
{\zihao{-4}\timesnewroman                % Keywords也使用Times New Roman
\setlength{\baselineskip}{20pt}          % 设置行距
\noindent\textbf{Keywords:} \enkeywords